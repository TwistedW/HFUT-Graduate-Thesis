% 模板设置

\newcommand{\privacy}[1][公开]{#1} % 密级
\newcommand{\type}[1][【设计或者论文】]{#1} % 类型,选填
\newcommand{\titleCna}[1][XXX]{#1} % 中文题目,默认最长12字符,查过部分换行到\titleCnb,我感觉大部分同学的题目长度都是超过12个字符,所以这部分我分为a,b两部分
\newcommand{\titleCnb}[1][XXX]{#1} % 中文题目,默认最长12字符,补充到a
\newcommand{\titleEn}[1][XXX]{#1} % 英文题目

\newcommand{\keywordsCn}[1][]{#1} % 中文关键字
\newcommand{\keywordsEn}[1][]{#1} % 英文关键字

\newcommand{\supervisor}[1][XXX\hspace{2em}教授]{#1} % 导师姓名
\newcommand{\studentID}[1][XXX]{#1} % 学号
\newcommand{\studentNameCn}[1][XXX]{#1} % 填写中文姓名
\newcommand{\studentNameEn}[1][XXX]{#1} % 填写英文姓名

\newcommand{\finishedYear}[1][\the\year]{#1} % 论文完成日期: 年
\newcommand{\finishedMonth}[1][\the\month]{#1} % 论文完成日期: 月
\newcommand{\finishedDay}[1][\the\day]{#1} % 论文完成日期: 日


\newcommand{\department}[1][计算机与信息学院]{#1} % 系名称
\newcommand{\major}[1][信号与信息处理]{#1} % 专业名称
\newcommand{\enrolmentYear}[1][【2017级】]{#1} % 入学年份
\newcommand{\object}[1][XXX]{#1} %研究方向



% 字号设置
\newcommand{\chuhao}{\fontsize{42pt}{\baselineskip}\selectfont}     % 字号设置
\newcommand{\xiaochuhao}{\fontsize{36pt}{\baselineskip}\selectfont} % 字号设置
\newcommand{\yichu}{\fontsize{32pt}{\baselineskip}\selectfont}      % 字号设置
\newcommand{\yihao}{\fontsize{28pt}{\baselineskip}\selectfont}      % 字号设置
\newcommand{\erhao}{\fontsize{21pt}{\baselineskip}\selectfont}      % 字号设置
\newcommand{\xiaoerhao}{\fontsize{18pt}{\baselineskip}\selectfont}  % 字号设置
\newcommand{\sanhao}{\fontsize{15.75pt}{\baselineskip}\selectfont}  % 字号设置
\newcommand{\sihao}{\fontsize{14pt}{\baselineskip}\selectfont}      % 字号设置
\newcommand{\xiaosihao}{\fontsize{12pt}{\baselineskip}\selectfont}  % 字号设置
\newcommand{\wuhao}{\fontsize{10.5pt}{\baselineskip}\selectfont}    % 字号设置
\newcommand{\xiaowuhao}{\fontsize{9pt}{\baselineskip}\selectfont}   % 字号设置
\newcommand{\liuhao}{\fontsize{7.875pt}{\baselineskip}\selectfont}  % 字号设置
\newcommand{\qihao}{\fontsize{5.25pt}{\baselineskip}\selectfont}    % 字号设置
\newcommand{\fighao}{\fontsize{11pt}{\baselineskip}\selectfont}    % 字号设置

% 下划线
\newcommand{\underlineFixlen}[2][3.5cm]{\underline{\makebox[#1][c]{#2}}}


% 中文摘要
\renewenvironment{abstract}{
%\thispagestyle{empty} % 去掉页码
{
\begin{center}
\Large \songti \bfseries 摘\hspace{1em}要\vspace{1.1cm}
\end{center}
}
\setlength{\parindent}{2em}
\setlength{\parskip}{0em}
\setlength{\lineskip}{0em} 
\setlength{\baselineskip}{20pt} % (宋体,小四;固定行距22磅,段前、段后均为0行间距。段落首行缩进2字符。)
\songti
}{
\setlength{\parindent}{0em}
\setlength{\parskip}{1em}
{\par \songti \bfseries{关键词:}}
\keywordsCn
\clearpage
}

% 英文摘要
\newenvironment{abstractEn}{
%\thispagestyle{empty} % 去掉页码
{
\begin{center}
\Large \bfseries ABSTRACT\vspace{1.5cm}
\end{center}
}
\setlength{\parindent}{2em}
\setlength{\parskip}{0em}
\setlength{\lineskip}{0em} 
\setlength{\baselineskip}{20pt} % 22磅行距,首行缩进1字符,段前、段后均为0行间距
}{
\setlength{\parindent}{0em}
\setlength{\parskip}{1em}
{\par \bfseries{KEYWORDS: }}
\keywordsEn
\clearpage
}

% 目录名
\renewcommand\contentsname{
\begin{center}
\songti \Large \bfseries 目\hspace{1em}录 % (宋体,小二号,加粗;居中,单倍行距,段前0.5行、段后1.5行间距)
\end{center}
\vspace{1em}
}
% 插图清单
\renewcommand\listfigurename{
\begin{center}
\songti \Large \bfseries 插图清单 % (宋体,小二号,加粗;居中,单倍行距,段前0.5行、段后1.5行间距)
\end{center}
\vspace{1em}
}

% 表格清单
\renewcommand\listtablename{
\begin{center}
\songti \Large \bfseries 表格清单 % (宋体,小二号,加粗;居中,单倍行距,段前0.5行、段后1.5行间距)
\end{center}
\vspace{1em}
}

\renewcommand\refname{\heiti \sanhao \bfseries 参考文献}


% 目录引线设置
\renewcommand{\cftdotsep}{1.5} % 线的密度
\renewcommand{\cftsecdotsep}{1.5} % section引线
\renewcommand{\cftsecleader}{\cftdotfill{\cftsecdotsep}}
\renewcommand{\cftsecpagefont}{}

% 插图清单
\renewcommand{\cftfigpresnum}{\figurename\enspace}

% 表格清单
\renewcommand{\cfttabpresnum}{\tablename\enspace}

% 致谢
\newenvironment{acknowledge}{
\clearpage
\vspace*{-2em}
%\phantomsection % 使得hyperref目录能够跳转到正确的位置
%\addcontentsline{toc}{section}{致谢} % 添加到目录中
\begin{center}
 \songti \Large \bfseries 致谢\end{center}\vspace{1.1cm}
\setlength{\parindent}{2em}
\setlength{\parskip}{0em}
\setlength{\lineskip}{0em} 
\setlength{\baselineskip}{20pt} % (宋体,小四;固定行距22磅,段前、段后均为0行间距。段落首行缩进2字符。)
\songti
\par
}{
	\par
	\hfill 作者:\studentNameCn

	\hfill \finishedYear\enspace 年\finishedMonth\enspace 月\finishedDay\enspace 日
}

% 攻读硕士学位期间的学术活动及成果情况
\newenvironment{achievement}{
\clearpage
\vspace*{-2em}
\phantomsection % 使得hyperref目录能够跳转到正确的位置
\addcontentsline{toc}{section}{攻读硕士学位期间的学术活动及成果情况} % 添加到目录中
\begin{center}
\songti \Large \bfseries 攻读硕士学位期间的学术活动及成果情况\end{center}\vspace{1.1cm}
\setlength{\parindent}{0em}
\setlength{\parskip}{0em}
\setlength{\lineskip}{0em} 
\setlength{\baselineskip}{20pt} % (宋体,小四;固定行距22磅,段前、段后均为0行间距。段落首行缩进2字符。)
\heiti
\par
}

% 附录
\renewenvironment{appendix}{
\clearpage
\vspace*{-2em}
\phantomsection % 使得hyperref目录能够跳转到正确的位置
\addcontentsline{toc}{section}{附录} % 添加到目录中
\begin{center}
 \songti \Large \bfseries 附录\end{center}\vspace{1.1cm}
\setlength{\parindent}{2em}
\setlength{\parskip}{0.5em}
\setlength{\baselineskip}{22pt} % 22磅行距,首行缩进1字符,段前、段后均为0行间距
\songti
\par
}{
}

% 图名称
\renewcommand{\figurename}{图}
\renewcommand{\tablename}{表}
